% Tretje poglavje pri predmetu, elementarna kombinatorika
\section{Elementarna kombinatorika}

\subsection{Izbori}

\subsubsection{Zgled}
Pri igri loto se v bobnu nahaja 39 kroglic, oštevilčenih s števili $1, \dots, 39$. Organizator igre iz bobna zaporedoma šestkrat izvleče po eno kroglico. Na koliko načinov lahko to stori? \\[1em]
Odgovor je odvisen od razlage besede "način":
\begin{itemize}
    \item ali naj kroglico, ki smo jo v posameznem koraku izvlekli vržemo v boben ali ne, in 
    \item ali je vrstni red izblečenih kroglic pomemben ali ne.
\end{itemize}
Dve osnovni lastnosti pri izbiri sta:
\begin{enumerate}
    \item \underline{ponavljanje}:
    \begin{itemize}
        \item izbori s ponavljanjem
        \item izbori brez ponavljanja
    \end{itemize}
    \item \underline{izbori}:
    \begin{itemize}
        \item urejeni izbori (varijacija)
        \item neurejeni izbori (kombinacija)
    \end{itemize}
\end{enumerate}

\subsection{Vrejeni izbori s ponavljanjem}
(izvlečene kroglice, v boben \underline{vračamo}, in vrstni red je \underline{pomemben})
\begin{align*}
    \underset{\text{k - terica}}{\underbrace{(1, 2, 1, 3, 12, 5)}}
\end{align*}

\subsubsection{Definicija}
Naj bo $N$ končna množica in $k \in \mathbb{N}$. Potem bomo urejeni k-terici $(a_1, \dots, a_k)$ elementi množice $N$ rečemo \underline{urejen izbor reda $k$ na množici $N$}. (lahko bi tukaj dodali da je \underline{s ponavljanjem}, ampak samo če želimo to poudariti, sicer bomo to izpustili) \\[1em]
Množico vseh takih izborov označimo z $\overline{V}(N, k)$.

\subsubsection{Zgled}
Recimo da imamo množico $N = \{a, b, c\}$ in je $k = 4$, potem lahko en tak izbor predstavimo kot $(a, c, c, b)$, in tak izbor lahko predstavimo tudi z funkcijo katera slika iz množice $1, \dots, k$ v $N$, brez kakršnih koli omejitev.
\begin{align*}
    (a, b, c, d) &\equiv
    \boxed{
        \begin{aligned}[c]
            &1 \to a \\
            &2 \to c \\
            &3 \to c \\
            &4 \to b
        \end{aligned}
    }
\end{align*}

\begin{opomba}
    Vzamemo k-terico $(a_1, \dots, a_k)$ elementa množice $N$ lahko razumemo tudi kot funkcijo ki slika iz množice $\{1, \dots, k\}$ v množico $N$.
\end{opomba}

\noindent
Ker množica $\overline{V}(N, k)$ vsebuje vse svoje n-terice. \\

\noindent
Od tod iz načela produkta sledi:

\subsubsection{Trditev}
Naj bo $N$ poljubna množica z $n$ elementi in $k \in \mathbb{N}$. Tedaj je v množici $\overline{V}(N, k)$ natanko $n^k$ izborov.
\begin{align*}
    |\overline{V}(N, k)| = |N^k| = |N|^k = n^k
\end{align*}

\subsubsection{Zgled}
Vsak vikend v februarju lahko obiščemo enega od 3 kinematografov. Koliko različnih zaporedij obiskov je možnih, pri čemer so ponovni obiski seveda dovoljeni? \\[1em]
Torej, imamo $K = \{K_1, K_2, K_3\}$ (kinematografov) in 4 vikenda, torej eden primer takega zaporedja obiskov bi bil npr. $(K_2, K_1, K_1, K_3)$

\begin{align*}
    |\overline{V}(K, 4)| = |K|^4 = 3^4 = 81
\end{align*}